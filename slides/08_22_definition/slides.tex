\documentclass[10pt]{beamer}

%%%
% PREAMBLE FOR THIS DOC 
%%%
%https://tex.stackexchange.com/questions/68821/is-it-possible-to-create-a-latex-preamble-header
\usepackage{/Users/miw267/Repos/csci246_spring2025/slides/preambles/beamer_preamble_for_CSCI246}


%%% TRY TO RESHOW TOC AT EACH SECTION START (with current section highlighted)
% Reference: https://tex.stackexchange.com/questions/280436/how-to-highlight-a-specific-section-in-beamer-toc
\newcommand\tocforsect[2]{%
  \begingroup
  \edef\safesection{\thesection}
  \setcounter{section}{#1}
  \tableofcontents[#2,currentsection]
  \setcounter{section}{\safesection}
  \endgroup
}


\usepackage[normalem]{ulem} % for strikeout (\sout)

%%%% HERES HOW TO DO IT CORRECTLY
% FIRST IN .STY FILE, DO
%\usetheme[sectionpage=none]{metropolis}
% THEN AT EACH SECTION DO
%\begin{frame}{Outline}
%  \tableofcontents[currentsection]	
%\end{frame}



%\setbeamertemplate{navigation symbols}{}
%\setbeamertemplate{footline}[frame number]{}


%%%
% DOCUMENT
%%%

\begin{document}

%\maketitle

%% Title page frame
%\begin{frame}
%    \titlepage 
%\end{frame}



\title{08/22/2025: Definition}
\author{CSCI 246: Discrete Structures}
\date{Textbook reference: Sec 3, Scheinerman}

\begin{frame}
    \titlepage 
\end{frame}

\begin{frame}
\begin{myyellowbox}[title=Today's Agenda]
\begin{itemize}
	\item Overview / Q \& A ($\approx$ 5 mins)
	\item Group exercises ($\approx$ 25 mins)
	\item Discussion ($\approx$ 20 minutes)
\end{itemize}


\end{myyellowbox}
\vfill 

\end{frame}





%
%\begin{frame}[standout]
%Q\&A On Previous Group Exercises
%\end{frame}

\begin{frame}{Integers}

\begin{mygreenbox}
\textbf{Definition.} The set of \textit{integers}, denoted $\mathbb{Z}$, is given by
\[\mathbb{Z} \defeq \set{\hdots, -3, -2, -1, 0, 1, 2, 3, \hdots}. \]
That is, the integers are the positive whole numbers, the negative whole numbers, and zero.
\end{mygreenbox}

\end{frame}


\begin{frame}[standout]
Group exercises
\end{frame}

\begin{frame}{Random group assignments}
\footnotesize 
\vfill 
\begin{columns}
\begin{column}{0.33\textwidth}
Aaron Christensen: 3 \\ 
Aidan Sinclair: 13 \\ 
Brendan Kelly: 1 \\ 
Buggy Garza: 6 \\ 
Cedric Jefferson: 17 \\ 
Conner Brost: 7 \\ 
Connor Graville: 3 \\ 
David Knauert: 1 \\ 
David Oswald: 4 \\ 
Elias Martin: 5 \\ 
Ericson O'Guinn: 8 \\ 
Erik Halverson: 7 \\ 
Francis Bush: 9 \\ 
Garrett Miller: 15 \\ 
George Cutler: 16 \\ 
Georgia Franks: 2 \\ 
Gregor Schmidt: 3 \\\end{column}
\begin{column}{0.33\textwidth}
Hakyla Riggs: 6 \\ 
Izayah Abayomi: 11 \\ 
Jacob Ketola: 10 \\ 
Jacob Ruiz: 17 \\ 
Jaden Hampton: 5 \\ 
Jeremy Ness: 10 \\ 
Jonah Day: 1 \\ 
Karter Gress: 8 \\ 
Kyle Hoerner: 10 \\ 
Landry Clarke: 4 \\ 
Leon BirdHat: 5 \\ 
Lillian Ziegler: 11 \\ 
Matthew Rau: 9 \\ 
Micah Miller: 14 \\ 
Michael Pitman: 15 \\ 
Nathan Campbell: 11 \\\end{column}
\begin{column}{0.33\textwidth}
Nathan Hooley: 15 \\ 
Nicholas Rugani: 8 \\ 
Noah Andersson: 14 \\ 
Olivia Greuter: 12 \\ 
Peter Van Vleet: 14 \\ 
Pierce Dotson: 9 \\ 
Quinn Carlson: 2 \\ 
Ridley Christoferson: 16 \\ 
Riley Smith: 6 \\ 
Sierra Holleman: 13 \\ 
Tanner Gramps: 7 \\ 
Timothy True: 13 \\ 
Titus Sykes: 12 \\ 
Trey Randall: 4 \\ 
William Grant: 16 \\ 
William Sheldon: 12 \\ 
Zachary Reller: 2 \\
\end{column}
\end{columns}
\vfill
\end{frame}

\begin{frame}{Group exercises}
\begin{enumerate}
\item  Please determine which of the following are true or false; use Definition 3.2 to explain your answers: (a) $3 \mid 100$, (b) $3 \mid 99$, (c) $-3 \mid 3$, (d) $-5 \mid -5$, (e) $-2 \mid -7$, (f) $0 \mid 4$, (g) $4 \mid 0$, (h) $0 \mid 0$.
\item None of the following is a prime.  Explain why they fail to satisfy Definition 3.5.  Which of the numbers is composite? (a) 21, (b) 0, (c) $\pi$, (d) $\half$, (e) -2, (f) -1.  
\item Define what it means for an integer to be a \textit{perfect square}.  For example, the integers 0,1,4,9, and 16 are perfect squares.  Your definition should begin: \\
\hspace{0.5cm} An integer $x$ is a \textit{perfect square} provided ...
\item Here is a possible alternative to Definition 3.2: We say that $a$ is \textit{divisible} by $b$ provided $\frac{a}{b}$ is an integer.  Explain why this alternative definition is different from Definition 3.2.  \\
\hspace{0.5cm} Here, \textit{different} means that the definitions specify \textit{different concepts}.  So to answer this question, you should find integers $a$ and $b$ such that $a$ is divisible by $b$ according to one definition, but $a$ is not divisible by $b$ according to the other definition. 
\end{enumerate}


\end{frame}

\begin{frame}{Solution to group exercise \#1}
%\textbf{Problem.} Please determine which of the following are true or false; use Definition 3.2 to explain your answers.

\begin{mygreenbox}
\textbf{Definition.} Let $a$ and $b$ be integers.  We say $a$ is \textit{divisible} by $b$, written $b \mid a$, provided there is an integer $c$ such that $bc=a$. 	
\end{mygreenbox}
\vspace{-.2cm}
\textbf{Solution.}
\begin{enumerate}[a.]
\item $3 \mid 100$? \pause False. There is no integer $c$ such that $3c=100$.  (Why? \pause  Note that there is exactly one number $c=\frac{100}{3}=33 \frac{1}{3}$ that satisfies the equation, but this $c$ is not an integer.) \pause 
\item $3 \mid 99$? \pause True. There is an integer $c=33$ such that $3c=99$. \pause 
\item $-3 \mid 3$? \pause True. There is an integer $c=-1$ such that $-3c=3$. \pause 
\item $-5 \mid -5$? \pause True. There is an integer $c=1$ such that $-5c=-5$. \pause 
\item $-2 \mid -7$? \pause False. There is no integer $c$ such that $-2c=-7$.   \footnotesize (The number $c=\frac{7}{2}$ uniquely satisfies the equation, but $c$ is not an integer.) \normalsize \pause 
\item $0 \mid 4$? \pause False. There is no integer $c$ such that $0c=4$. (Why?  \pause Note that $0c=0$ for all $c$.) \pause  
\item $4 \mid 0$? \pause True. There is an integer $c=0$ such that $4c=0$. \pause  
\item $0 \mid 0$? \pause  True. There is an integer $c$ such that $0c=0$. In fact, there are \textit{infinitely} many integer-valued solutions!
\end{enumerate}

\end{frame}


\begin{frame}{Solution to group exercise \#2 (first part)}
\textbf{Problem.} None of the following numbers is prime. Explain why they fail to satisfy Definition 3.5.
\vfill 

\begin{mygreenbox}
\textbf{Definition.} An integer $p$ is called \textit{prime} provided that $p>1$ and the only positive divisors of $p$ are $p$ and $1$.	
\end{mygreenbox}

\vfill 
\textbf{Solution.}
\begin{enumerate}[a.]
\item 21? \pause This is not prime since $3 \mid 21$, and $3 \not\in \set{1,21}$. \pause 
\item 0? \pause This is not prime since $0<1$. \pause 
\item $\pi$? \pause This is not prime since $\pi \not\in \mathbb{Z}$. \pause 
\item $\half$? \pause This is not prime since $\half \not\in \mathbb{Z}$. \pause 
\item -2? \pause This is not prime since $-2<1$. \pause 
\item -1? \pause This is not prime since $-1<1$. \pause 
\end{enumerate}
\vfill \vfill \vfill 
\textbf{Remark.} In the above, I used the shorthand notation $x \not\in A$, which means that $x$ is not a member of the set $A$. 
\end{frame}


\begin{frame}{Solution to group exercise \#2 (second part)}
\textbf{Problem.} None of the following numbers is prime. Which is composite?
\vfill 

\begin{mygreenbox}
\textbf{Definition.} An integer $a$ is called \textit{composite} provided that there is an integer $b$ such that $1<b<a$ and $b \mid a$.	
\end{mygreenbox}

\vfill 
\textbf{Solution.}
\begin{enumerate}[a.]
\item 21? \pause Composite, since $3 \mid 21$, and $1<3<21$. \pause 
\item 0? \pause Not composite, since $0<1$. \pause 
\item $\pi$? \pause Not composite, since $\pi \not\in \mathbb{Z}$. \pause 
\item $\half$? \pause Not composite, since $\half \not\in \mathbb{Z}$. \pause 
\item -2? \pause Not composite, since $-2 < 1$. \pause 
\item -1? \pause Not composite, since $-1 < 1$. \pause 
\end{enumerate}
\vfill 
\textbf{Remark.} Note from the definition that the integer $a$ can be composite only if $a>1$. We use this fact to answer parts b, e, and f.
\end{frame}

\begin{frame}{Solution to group exercise \#3}

\textbf{Problem.} Define what it means for an integer to be a \textit{perfect square}.  For example, the integers 0,1,4,9, and 16 are perfect squares.  

\pause 
\vfill 
\textbf{Solution.} An integer $x$ is a \textit{perfect square} provided there is an integer $y$ such that $x=y^2$.

\end{frame}

\begin{frame}{Solution to group exercise \#4}

\textbf{Problem.} Here is a possible alternative to Definition 3.2: We say that $a$ is \textit{divisible} by $b$ provided $\frac{a}{b}$ is an integer.  Explain why this alternative definition is different from Definition 3.2.  \\
\hspace{0.5cm} Here, \textit{different} means that the definitions specify \textit{different concepts}.  So to answer this question, you should find integers $a$ and $b$ such that $a$ is divisible by $b$ according to one definition, but $a$ is not divisible by $b$ according to the other definition.

\pause 
\vfill 
\textbf{Solution.} Consider $a=b=0$. Then,  $a \mid b$ according to Definition 3.2 (as we discovered in group exercise \#1h). However, $\frac{a}{b}$ is not an integer.
\end{frame}


\end{document}
